%%=============================================================================
%% Voorwoord
%%=============================================================================

\chapter*{\IfLanguageName{dutch}{Woord vooraf}{Preface}}
\label{ch:voorwoord}

%% TODO:
%% Het voorwoord is het enige deel van de bachelorproef waar je vanuit je
%% eigen standpunt (``ik-vorm'') mag schrijven. Je kan hier bv. motiveren
%% waarom jij het onderwerp wil bespreken.
%% Vergeet ook niet te bedanken wie je geholpen/gesteund/... heeft
Een bachelorproef maken over een onderwerp waardoor je zelf gepassioneerd bent, 
ik denk dat iedere student hier wel van droomt. Al reeds een lange tijd hou ik me bezig met 
het Flutter framework. Het was snel duidelijk dat Flutter een gewaagde concurrent zou worden voor andere cross-platform frameworks.

Het grootste vraagstuk in het Flutter framework is op de dag van vandaag nog steeds State Management.
Dit wetenschappelijk onderzoek zal zowel een meerwaarde bieden op mijn vraagstuk alsook voor de Flutter community.

Ik ben in contact gekomen met Flutter toen het nog in beta was. Waarom ik Flutter op de voet ben blijven volgen kan ik kort beargumenteren. Het feit dat Google achter het Flutter framework zit, zorgt voor zekerheid op vlak van financiële middelen. Daarnaast zorgt Flutter ervoor dat het ontwikkelen van applicaties zeer toegankelijk is geworden, dit wekt heel wat interesse op bij nieuwe ontwikkelaars. In de beginperiode van Flutter was er weinig tot geen documentatie. Dit resulteert in het feit dat er zelf veel onderzocht moest worden.

Initieel was ik van plan om het in mijn bachlorproef te hebben over een inleiding tot het Flutter framework, maar dit is reeds uitvoerig gedaan door Bram De Coninck, tevens mijn co-promotor. Ik ben met Bram in contact gekomen door de Flutter community, beiden waren we aanwezig op gemeenschappelijke fora. Aangezien ik ook zelf een onderzoek wou starten naar State Management in Flutter heb besloten om dit als bachelorproefonderwerp te nemen.


Verder wil ik mijn promotor, Antonia Pierreux, die mij goed heeft begeleid tijdens de bachelorproefperiode, bedanken.
Daarnaast wil ik mijn co-promotor, Bram De Coninck, bedanken. Hij is degene die ervoor gezorgd heeft dat deze bachelorproef kan doorgaan. Ik hoop van harte dat wij nog een lange tijd in contact met elkaar zullen staan. Ik wil ook een vriend en collega Robin Wijnant bedanken voor zijn feedback tijdens het maken van deze bachelorproef.

Tevens wil ik ook de Flutter community bedanken die erop staat dat er veel wordt gedocumenteerd, in bijzonder Didier Boelens die openstond voor het beantwoorden van meer technische vragen.

Deze bachelorproef heeft als bedoeling een aanvulling te zijn voor de jonge en actieve Flutter community bij het kiezen van een State Management in een Flutter applicatie. Tijdens het maken van deze bachelorproef zijn er vaak wijzigingen toegepast aangezien de Flutter ontwikkelaars druk bezig zijn met het bijwerken van het Flutter framework. Zo is op heden de nieuwe Dart SDK versie \verb|2.7| en is de stabiele versie van Flutter reeds \verb|1.12.13-hotfix.5|.