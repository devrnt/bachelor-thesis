%%=============================================================================
%% Voorwoord
%%=============================================================================

\chapter*{\IfLanguageName{dutch}{Woord vooraf}{Preface}}
\label{ch:voorwoord}

%% TODO:
%% Het voorwoord is het enige deel van de bachelorproef waar je vanuit je
%% eigen standpunt (``ik-vorm'') mag schrijven. Je kan hier bv. motiveren
%% waarom jij het onderwerp wil bespreken.
%% Vergeet ook niet te bedanken wie je geholpen/gesteund/... heeft
Een bachelorproef maken over een onderwerp waardoor je zelf gepassioneerd bent, 
ik denk dat elke student hiervan droomt. Al reeds een lange tijd ben ik bezig met 
het Flutter framework, tijdens deze periode zijn er natuurlijk gebreken naar boven gekomen.

Het grootste vraagstuk in het Flutter framework is op de dag van vandaag nog steeds State Management.
Dit wetenschappelijk onderzoek zal zowel een antwoord bieden op mijn vraagstuk en zowel voor de Flutter community.
Bij dit topic worden zeer veel vragen gesteld bij Flutter nieuwkomers en dit onderzoek zal een antwoord bieden.

Toen Flutter nog in beta versie was ben ik ermee in contact gekomen, waarom ik Flutter op de voet ben blijven volgen kan ik kort argumenteren. Het feit dat Google achter het Flutter team staat zorgt ervoor dat er een zekerheid is in verband met financiële middelen. Daarnaast zorgt Flutter ervoor dat het ontwikkelen van applicatie zeer toegankelijk is geworden. In deze periode was er weinig documentatie, videomateriaal... Dit resulateert op het feit dat er zelf veel onderzoek moest uitgevoerd worden.

Initieel was ik van plan om het in mijn bachlorproef te hebben over het Flutter framework, maar dit is reeds uitvoerig gedaan door Bram De Coninck, waarmee ik ook in contact ben gekomen door de Flutter-community. Aangezien ik ook zelf een onderzoek wou starten naar State Management in Flutter heb besloten om dit als bachelorproefonderwerp te nemen.


Verder wil ik mijn promotor, Antonia Pierreux, die mij goed heeft begeleid tijdens de bachelorproefperiode, bedanken.
Daarnaast wil ik mijn co-promotor, Bram De Coninck, bedanken. Hij is degene die ervoor gezorgd heeft dat deze bachelorproef kan doorgaan. Ik hoop van harte dat wij nog een lange tijd in contact met elkaar zullen staan.

Tevens wil ik ook de Flutter community bedanken die erop staat dat er veel wordt gedocumenteerd, in bijzonder Didier Boelens die openstond voor het beantwoorden van meer technische vragen.
