%%=============================================================================
%% Samenvatting
%%=============================================================================

% TODO: De "abstract" of samenvatting is een kernachtige (~ 1 blz. voor een
% thesis) synthese van het document.
%
% Deze aspecten moeten zeker aan bod komen:
% - Context: waarom is dit werk belangrijk?
% - Nood: waarom moest dit onderzocht worden?
% - Taak: wat heb je precies gedaan?
% - Object: wat staat in dit document geschreven?
% - Resultaat: wat was het resultaat?
% - Conclusie: wat is/zijn de belangrijkste conclusie(s)?
% - Perspectief: blijven er nog vragen open die in de toekomst nog kunnen
%    onderzocht worden? Wat is een mogelijk vervolg voor jouw onderzoek?
%
% LET OP! Een samenvatting is GEEN voorwoord!

\IfLanguageName{english}{%
\selectlanguage{dutch}
\chapter*{Samenvatting}
\selectlanguage{english}
}{}

%%---------- Samenvatting -----------------------------------------------------
% De samenvatting in de hoofdtaal van het document

\chapter*{\IfLanguageName{dutch}{Samenvatting}{Abstract}}
De toegankelijkheid van Flutter zorgt voor veel aandacht van ontwikkelaars. Bij het ontwikkelen van een applicatie komt er meer werk aan te pas dan enkel de visuele aspecten, zo komt State Management in elke applicatie terug. Dat State Management een veelomvattend begrip is wordt beaamd door tal van ontwikkelaars. Daarom heeft het team dat Flutter ontwikkelt, een pagina gewijd aan State Management op de officiële site van Flutter. Op deze pagina worden enkele State Management benaderingen opgelijst en wordt de keuze aan de ontwikkelaar overgelaten enkel op basis van voorkeur.\newline 

Deze bachelorproef toont aan hoeveel impact een State Management benadering met zich teweegbrengt. Dit onderzoek probeert extra motivatiepunten voor te leggen om een bepaalde benadering al dan niet te hanteren. Er wordt onderzoek gevoerd naar het aantal lijnen code en de prestaties per State Management benadering. Voor de prestaties wordt er gekeken naar de CPU-tijd en de overgeslagen frames.
\newline
Vooraleer dit onderzoek uitgevoerd wordt, wordt er een literatuurstudie gemaakt. In deze literatuurstudie wordt de huidige stand van zaken in verband Flutter kort besproken. Ook worden de huidige populaire State Management benaderingen uitvoerig besproken. Deze besproken benaderingen worden daarna toegepast in het experiment. \newline 

De resultaten van het experiment duiden op het feit dat bepaalde benaderingen aantrekkelijker zijn. Dit zowel op basis van het aantal lijnen code als de prestaties. Over het algemeen kunnen er voorkeuren gekozen worden op basis van de resultaten van dit onderzoek. Sommige benaderingen gebruiken een andere architectuur die de nodige inwerkingstijd vergen, maar zorgt voor een gestructureerde architectuur. De keuze hangt ook af van de voorkeur van de ontwikkelaar. Dit is en blijft louter een subjectieve kwestie.\newline

Dat Flutter een grote pion is in cross-platform development is reeds geweten. Daarom is het nuttig om onderzoeken uit te voeren gelijkaardig aan deze bachelorproef. Dit onderzoek kan de basis vormen voor verdere onderzoeken, want die zijn zeker mogelijk. Zo is dit onderzoek uitbreidbaar met extra meetstaven voor grondiger onderzoek.
